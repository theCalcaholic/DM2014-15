\documentclass[11pt, oneside]{scrartcl}   	% use "amsart" instead of "article" for AMSLaTeX format
\usepackage[T1]{fontenc}
\usepackage[utf8]{inputenc}
\usepackage[bottom=6em]{geometry}
\geometry{a4paper}                   		% ... or a4paper or a5paper or ... 
\usepackage{titling}
\usepackage{dot2texi}
\usepackage{tikz}
\usepackage{shapes,arrows}

\setlength{\droptitle}{-10em}   % This is your set screw
\setlength{\footskip}{4em}


%\geometry{landscape}                		% Activate for rotated page geometry
%\usepackage[parfill]{parskip}    		% Activate to begin paragraphs with an empty line rather than an indent
%\usepackage{graphicx}				% Use pdf, png, jpg, or eps§ with pdflatex; use eps in DVI mode
								% TeX will automatically convert eps --> pdf in pdflatex		
\usepackage{amssymb}
\usepackage{german}
%\usepackage{tikz}
%\usetikzlibrary{arrows}

\usepackage{fancyhdr}
\pagestyle{fancy}

%SetFonts

%SetFonts

% meine Hilfstabelle
% ------------------------
% \neg - nicht
% \vee - oder
% \wedge - und
% \in - ist element von
% \exists - es existiert
% \forall - drei alle
% \subset / \subseteq - Teilmenge von
% \supset / \supseteq - umgekehrt
% \rightarrow - Implikation
% \leftrightarrow - "aquivalenz
% -- die beiden oben mit R und L (großen Anfangsbuchstaben) auch als Doppelpfeile
% \mapsto - Zuweisung
%
% Syntax: A $\leftrightarrow$ B oder  so (mit $ Symbolen klammern!)


\title{Mathematik I}
\subtitle{Hausaufgaben zum 20. und 21. November 2014}
\author{Thomas Vogt, 6687486 \and Tobias Knöppler, 6523815}
\date{}

\lhead{Vogt / Knöppler}
\rhead{6687486 / 6523815}


\begin{document}
\maketitle
\section*{Aufgabe 1}

\subsection*{a}
(dot)

\subsection*{b}
Um Reflexivität zu erreichen:
(a,a), (b,b), (c,c), (d,d), (e,e), (f,f)

Und um Transitivität zu erreichen:
(a,c), (a,d), (b,d), (a,f), (b,e), (a,e)

\subsection*{c}

\subsection*{d}
Da wir keine Kanten entfernen und alle Knoten miteinander zusammenhängen, gibt es nur eine Äquivalenzklasse, in der alle Knoten miteinander verbunden sind, die Paare sind also:

(a,a), (a,b), (a,c), ... (a,f), (b,a), (b,b), ..., (f,a), (f,b), (f,c), (f,d), (f,f)


\section*{Aufgabe 2}

\subsection*{a}
(dot)

\subsection*{b}
Alle reflexiven Paare (a,a), (b,b), (c,c), (d,d), (e,e) und (f,f) sowie das Paar (b,d) um Transitivität herzustellen.

\subsection*{c}
(dot)

\subsection*{d}
Alle aus (b) sowie die symmetrischen Paare (a,b), (d,a), (d,b), (f,e)

\section*{Aufgabe 3}

\subsection*{a}
Alle reflexiven Paare (a,a), (b,b), (c,c) und (d,d), alle symmetrischen Paare (b,a) und (d,c) sowie, um die Transitivität zu unterbinden, (b,c) und (c,b).

\subsection*{b}
Nur die reflexiven Paare (s.o.) hinzufügen, denn transitiv ist R schon und symmetrisch ist R nicht.

\subsection*{c}
Nur die symmetrischen Paare (s.o.) hinzufügen, denn transitiv ist R schon und reflexiv ist R nicht.

\section*{Aufgabe 4}
(1,1), (1,2), (1,3), (1,4), (1,6), (1,8), (1,12, (1,24),\\
(2,2), (2,4), (2,6), (2,8), (2,12), (2,24),\\
(3,3), (3,6), (3,12), (3,24),\\
(4,4), (4,8), (4,12), (4,24),\\
(6,6), (6,12), (6,24),\\
(8,8), (8,24),\\
(12,12), (12,24),\\
(24,24)

(dot)

\section*{Aufgabe 5}

\subsection*{a}
\begin{dot2tex}[dot,options=-tmath --autosize --cache]
\digraph G {
rankdir = LR
node[shape="none"]; start;
node [shape="box", color="blue"]; q0;
node [shape="circle",color="black"]; q_1 q_2 q_3;

start -> q_0;
q_0 -> q_1 [label="1"];
q_1 -> q_0 [label="1"];

q_1 -> q_2 [label="0"];
q_2 -> q_1 [label="0"];

q_2 -> q_3 [label="1"];
q_3 -> q_2 [label="1"];

q_3 -> q_0 [label="0"];
q_0 -> q_3 [label="0"];

%{} -> {1};
%{} -> {2};
%{} -> {1,2};
%{1} -> {1,2};
%{2} -> {1,2};
}
\end{dot2tex}


\subsection*{b}
ORDNUNG! (von unten nach oben wachsen, 0 ist unten!)
graph {
{} -- {1};
{} -- {2};
{} -- {3};
{1} -- {1,2};
{1} -- {1,3};
{2} -- {1,2};
{2} -- {2,3};
{3} -- {1,3};
{3} -- {2,3};
{1,2} -- {1,2,3};
{1,3} -- {1,2,3};
{2,3} -- {1,2,3};
}

\end{document}
