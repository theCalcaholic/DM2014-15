\documentclass[fleqn]{article}
\usepackage[utf8]{inputenc}
\usepackage{geometry}
\usepackage{fancyhdr}
\usepackage{amsmath,amsthm,amssymb}
\usepackage{graphicx}
\usepackage{hyperref}
\usepackage{lipsum}
\usepackage{ulem}
\usepackage{comment}
\usepackage{enumerate}
\usepackage{titlesec}

% custom commands
\newcommand{\leadingzero}[1]{\ifnum #1<10 0#1\else#1\fi}
\newcommand{\gerdate}[3]{\leadingzero{#1}.\leadingzero{#2}.\leadingzero{#3}}
\newcommand{\gertoday}{\gerdate{\the\day}{\the\month}{\the\year}}
\newcommand{\R}{\mathbb{R}}
\newcommand{\N}{\mathbb{N}}
\newcommand{\Q}{\mathbb{Q}}

\setcounter{section}{0}
\setcounter{subsection}{0}
\pagestyle{fancy}

\lhead{Tobias Knöppler}
\chead{}
\rhead{\gertoday}
\lfoot{}
\cfoot{\thepage}
\rfoot{}
\setlength{\mathindent}{0pt}

% document specific settings
\renewcommand{\thesection}{}
\renewcommand{\thesubsection}{\arabic{section}. \alph{subsection})}
\renewcommand{\thesubsubsection}{\roman{subsubsection})}
\titleformat{\subsubsection}[runin]{\normalfont\normalsize\bfseries}{\thesubsubsection}{1em}{}

\title{ALA - Hausaufgaben zum 8. Mai 2014 (Blatt 04)}
\author{}
\date{\gertoday}
\begin{document}
\maketitle

\section{}
für $n = 0$: $\{\emptyset\}$; Die Menge der Teilmengen von dieser Menge ist: $\{\{\emptyset\}\}$ und besitzt damit $1 = 2^0$ Teilmengen\\ \\

\textbf{Induktionsanfang:} Es existiert ein n, für das jede n-elementige Menge genau $2^n$ Teilmengen hat.\\
\emph{Beweis:} für $M_{n_0} = 1$: $\{a\}$; Die Menge der Teilmengen von dieser Menge ist: $\{\{\emptyset\},\{a\}\}$ und besitzt damit $2 = 2^1$ Teilmengen.\\
\textbf{Induktionsschritt:}\\
\emph{Induktionsannahme:} $\forall (M_n)_{n>n_0} (|\{x|x \subset M_n\}|  = 2^n \rightarrow |\{y|y \subset M_{n+1}\}| = 2^{n+1})$\\
\emph{Beweis:}
\begin{flalign}
    |M_n| = \sum\limits_{i = 1}^{n} = (\matrix{n##i)
\end{flalign}

\end{document}
