\documentclass[fleqn]{article}
\usepackage[utf8]{inputenc}
\usepackage{geometry}
\usepackage{fancyhdr}
\usepackage{amsmath,amsthm,amssymb}
\usepackage{graphicx}
\usepackage{hyperref}
\usepackage{lipsum}
\usepackage{ulem}
\usepackage{comment}
\usepackage{enumerate}
\usepackage{titlesec}
\usepackage{ wasysym }

% custom commands
\newcommand{\leadingzero}[1]{\ifnum #1<10 0#1\else#1\fi}
\newcommand{\gerdate}[3]{\leadingzero{#1}.\leadingzero{#2}.\leadingzero{#3}}
\newcommand{\gertoday}{\gerdate{\the\day}{\the\month}{\the\year}}
\newcommand{\R}{\mathbb{R}}
\newcommand{\N}{\mathbb{N}}
\newcommand{\Q}{\mathbb{Q}}
\newcommand{\Z}{\mathbb{Z}}

\setcounter{section}{0}
\setcounter{subsection}{0}
\pagestyle{fancy}

\lhead{Tobias Knöppler}
\chead{}
\rhead{\gertoday}
\lfoot{}
\cfoot{\thepage}
\rfoot{}
\setlength{\mathindent}{0pt}

% document specific settings
\renewcommand{\thesubsection}{\arabic{section}. \alph{subsection})}
\renewcommand{\thesubsubsection}{\roman{subsubsection})}
\titleformat{\subsubsection}[runin]{\normalfont\normalsize\bfseries}{\thesubsubsection}{1em}{}

\title{DM - Hausaufgaben zum 31. Oktober 2014 (Blatt 04)}
\author{}
\date{\gertoday}
\begin{document}
\maketitle

\section{}
\subsection{}
\subsubsection{}
Es gibt keine injektive Funktion $f: A \mapsto B$, da $\mid B \mid > \mid A \mid$.
\subsubsection{}
Ja, z.B.: 
\begin{align*}
    f: f(1) &= a\\
    f(2) &= b\\
    f(3) &= c
\end{align*}
\subsubsection{}
Nein, da es keine injektive Funktion $A \mapsto B$ gibt, existiert auch keine bijektive Funktion $A \mapsto B$.
\subsection{}
\subsubsection{}
Nein, da eine injektive Abbildung $A \mapsto B$ mit $\mid A \mid = \mid B \mid$ immer auch bijektiv sein muss.
\subsubsection{}
Ja, z.B. 
\begin{flalign*}
f: f(1) &= a\\
f(2) &= b,c\\
f(3) &= c
\end{flalign*}
\subsubsection{}
Ja, z.B.
\begin{flalign*}
    f: f(1) &= a\\
    f(2) &= b\\
    f(3) &= c
\end{flalign*}

\subsection{}
\subsubsection{}
Ja, z.B.
\begin{flalign*}
    f: f(1) &= a\\
    f(2) &= b\\
    f(3) &= c
\end{flalign*}
\subsubsection{}
Ja, z.B.
\begin{flalign*}
    f: f(1) &= a,b\\
    f(2) &= b,c\\
    f(3) &= d
\end{flalign*}
\subsubsection{}
Ja, z.B.
\begin{flalign*}
    f: f(1) &= a,b\\
    f(2) &= c\\
    f(3) &= d
\end{flalign*}
\section{}
\subsection{}
f ist injektiv, da $\forall n (n \in \Z \Rightarrow 2n \in \Z)$.\\
Beweis:\\
\begin{enumerate}[I]
\item Induktionsannahme: $ \exists n \in \Z (2n \in \Z)$\\
    Beweis: $2 \cdot 1 = 2 \in \Z$
\item Induktionsschritt:\\
    Behauptung: $2n \in \Z \Rightarrow 2(n+1) \in \Z$\\
    Beweis:
    \[
        2(x+1) = 2x +2
    \]
    \[
        (2 \in \Z) \wedge (2x \in \Z) \Rightarrow 2x + 2 \in \Z ~\square
    \]
\end{enumerate}
f ist nicht surjektiv, da es z.B. kein n gibt, für das gilt f(n) = 5.\\
Beweis:\\
Angenommen, es gelte: $\exists n (f(n)=5 \Rightarrow f(n) = 2 \cdot \frac{5}{2})$, dann wäre $n = \frac{5}{2} \notin \Z$, womit die Annahme, dass f surjektiv sei, zum Widerspruch geführt ist. $\square$\\\\
f ist nicht bijektiv, da f nicht surjektiv ist.
\subsection{}
g ist injektiv, da $\forall n \in \Z (2n + 5 \in \Z)$.\\
Beweis:\\
\begin{enumerate}[I]
    \item Induktionsannahme (IA): $\exists n \in \Z (2n + 5 \in \Z)\\
        (n=1 \Leftrightarrow 2n + 5 = 7) \wedge (7 \in \Z) \Rightarrow \text{IA ist erfüllt für n = 1.}$
    \item Induktionsschritt:\\
        Behautptung: $(2n + 5 \in \Z) \Rightarrow (2(x+1) + 5 \in \Z)$\\
        Beweis:
        \[
            2(n+1) + 5 = 2n +7 = 2n + 5 + 2
        \]\[
            (2n + 5 \in \Z) \wedge (2 \in \Z) \Rightarrow 2n + 7 \in \Z
        \]
    \item Induktionsschluss:\\
        $(I) \wedge (II) \Rightarrow \forall n \in \Z (2n + 5 \in \Z)$\\
        Daraus folgt, dass g injektiv ist. $\square$
\end{enumerate}

g ist nicht surjektiv, da $\exists n \notin \Z (2n + 5 \in \Z)$.\\
Beweis:\\
Angenommen, $2n + 5 = 6$.\\
Dann gälte $n = \frac{6-5}{2} = \frac{1}{2} \notin \Z \Rightarrow \exists n \notin \Z (2n + 5 \in \Z)~ \square$
\\
g ist nicht bijektiv, da g nicht surjektiv ist.

\subsection{}
h ist nicht injektiv, da h(-2) = 9 = h(2).\\
\\
h ist nicht surjektiv, da $n \notin \Z \text{ für } (2n^2 + 5 = 2)$.\\
Beweis:\\
$n^2 + 5 = 2 \Leftrightarrow n^2 = -3 \Leftrightarrow n = \sqrt{3} \Rightarrow n \notin \Z~\square$.\\
\\
h ist nicht bijektiv, da h weder injektiv noch surjektiv ist.
\section{}
\subsection{}
Behauptung: f ist injektiv.\\
Beweis:\\
Angenommen, f wäre nicht injektiv. Dann gäbe es $m, n \in \Z$ mit $m \neq n$, für die gälte:\\
\begin{flalign*}
f(m) &= f(n)\\
\Rightarrow &\text{I}~~ m^2 - 5 = n^2 - 5 \Leftrightarrow m = \pm \sqrt{n^2} = \pm n =* -n\\
&\text{II}~~ (m^2 - 2)^2 = (n - 2)^2\\
&\text{m in II}~~ (-n - 2)^2 = (n - 2)^2\\
&~~ n^2 + 4n + 4 = n^2 - 4n + 4\\
&\qquad 4n = - 4n \Leftrightarrow 4 = -4 ~\text{\lightning}\\
\\
&\text{* gilt wegen der Annahme, dass m $\neq$ n seien.}
\end{flalign*}
Damit ist die Annahme, es gäbe $m, n \in \Z$ mit $m \neq n$, für die gilt: $f(m) = f(n)$ zum Widerspruch geführt und bewiesen, dass f injektiv ist. $\square$\\
\\
f ist nicht surjektiv, da es z.B. kein $m \in \Z$ gibt, für das gilt: $f(m) = (-7,4)$.\\
Beweis:\\
Angenommen, es gäbe ein $m \in \Z$, für das gilt: $f(m) = (-7, 4)$.\\
Dann gälte:\\
I  $m^2 - 5 = -7 \Leftrightarrow m = \pm \sqrt{-2}$\\
II  $(m-2)^2 = 4 \Leftrightarrow m = 4$\\
Da $m \neq 4 \sqrt{-2}$, ist die Annahme widerlegt, es gäbe ein $m \in \Z$ mit $f(m) = (-7, 4)$ und bewiesen, dass f nicht surjektiv ist. $\square$\\
\\
f ist nicht bijektiv, da f nicht surjektiv ist.

\subsection{}
g ist nicht injektiv, da $g(2,2) = 10 = g(2,-2)$.\\
\\
g ist surjektiv, da gilt: $\forall a \in \Z \exists n (g(0,n) = a)$.\\
Beweis:\\
Sei $m=0$. Dann gilt für $f(m,n) = -n$.\\
Wegen $(n \in \Z \Rightarrow -n \in \Z)$ folgt somit $\forall a \in \Z \exists m \exists n (f(m,n) = a)$.\\
Damit ist g surjektiv. $\square$\\
\\
g ist nicht bijektiv, da g nicht injektiv ist.

\subsection{}
h ist nicht surjektiv, da es keine $m,n \in \Z$ gibt, für die gilt: $f(m,n) = (2,-1)$.\\
Beweis:\\
Angenommen, es gäbe, $m,n \in \Z$, sodass $h(m,n) = (2,1)$, dann gälte:\\
I   $3m - n = 2 \Leftrightarrow m = \frac{2-n}{3}$\\
II  $-3m + n = 1$\\
m in II  $-3 \cdot \frac{2-n}{3} + n = 1 \Leftrightarrow -2 + n + n = 1 \Leftrightarrow n = 1$\\
n in I  $\frac{2-1}{3} = \frac{1}{3} \notin \Z \text{lightning}$\\
Damit ist die Annahme, es existierten $m,n$, sodass $h(m,n) = (2,1)$, zum Widerspruch geführt und bewiesen, dass h nicht surjektiv ist.\\
\\
h ist nicht bijektiv, da h weder injektiv noch surjektiv ist.

\section{}
Behauptung: $\forall n \in \N (\sum\limits_{i=1}^{n} (2i-1) = n^2)$\\

\begin{enumerate}[I]
\item Induktionsanfang:\\
    Behauptung: $\exists n (\sum\limits_{i=1}^{n}(2i-1) = n^2)$\\
    Beweis: Es sei $n = 1$. Dann ist $\sum\limits_{i=1}^{i}(2i-1 = 2 \cdot 1 - 1 = (n+1)^2$. $\square$\\
\item Induktionsschritt:\\
    Induktionsannahme: $(\sum\limits_{i=1}^{n} (2i - 1) = n^2 \Rightarrow \sum\limits_{i = 1}^{n + 1}(2i-1) = (n+1)^2)$\\
    Beweis:\\
    $\sum\limits_{i=1}^{n+1}(2i-1) \Leftrightarrow 2(n+1) - 1 + \sum\limits_{i=1}^{n}(2i-1) \Leftrightarrow 2n + 1 + \sum\limits_{i=1}^{n}(2i-1) = (n+1)^2 = n^2 + 2n + 1 = n^2 + 2n + 1$\\
    $\Leftrightarrow \sum\limits{i=1}{n}(2i-1) = n^2$ $\square$.
\end{enumerate}
\end{document} 
